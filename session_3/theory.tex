\documentclass{beamer}
\usepackage{amsmath}
\usepackage{graphicx}
\graphicspath{{./figures/}}

\usepackage{tikz}
\usepackage{pgfplots}
\pgfplotsset{compat=1.18}

\usetheme{Boadilla}
\usecolortheme{seahorse}

\title{Probabilistic and Statistical Modeling}
\author{Vasilis Gkolemis}
\institute{ATHENA RC | HUA}
\date{June 2025}

\begin{document}



\begin{frame}
  \titlepage
  \vfill
  \footnotesize
  \textcopyright\
  Vasilis Gkolemis, 2025. Licensed under CC BY 4.0.
\end{frame}

\begin{frame}{Contents}
  \tableofcontents
\end{frame}


\begin{frame}{Helping Material}
  \begin{itemize}
    \item \textbf{Primer on Probabilistic Modeling} \url{https://www.inf.ed.ac.uk/teaching/courses/pmr/22-23/assets/notes/probabilistic-modelling-primer.pdf}
  \end{itemize}
\end{frame}


\section{Recap}

\begin{frame}{Session 1 – Recap}
  \begin{itemize}
    \item \textbf{What we covered:}
    \begin{itemize}
      \item \textbf{Probabilistic Modeling:}
      Model real-world phenomena using probabilities.

      \item \textbf{Probabilistic Reasoning (Inference):}
      Use known probabilities to infer unknowns.

      \item \textbf{Bayesian Analysis:}
      Modeling and Reasoning with Bayes’ rule.

      \item \textbf{Core Rules of Probability:}
        The sum, product and Bayes’ rule.

      \item \textit{Example: Alzheimer’s diagnostic test.}
    \end{itemize}

    \item \textbf{What’s still to explore:}
    \begin{itemize}
      \item Our example was simple; two $1D$ random variables were enough:
      \begin{itemize}
        \item $X$: the test result
        \item $Y$: the disease status
      \end{itemize}
    \item Real-world problems are much more complex
      \begin{itemize}
      \item Involve more random variables and of higher dimensions.
      \end{itemize}
    \end{itemize}
  \end{itemize}
\end{frame}

\begin{frame}{Session 2 – Recap}
  \begin{itemize}
    \item \textbf{What we covered:}
    \begin{itemize}
      \item \textbf{Multivariate Random Variables and Distributions:}
      \begin{itemize}
        \item PDFs, PMFs and CDFs
        \item Key properties: expectation and variance.
        \item How to sample from these distributions.
        \item Key-distributions: Bernoulli, Normal, Poisson.
      \end{itemize}

      \item We now have powerful tools to model complexity!
    \end{itemize}

    \item \textbf{What’s still to explore:}
    \begin{itemize}
    \item How to use RVs and their properties to draw conclusions about real-world phenomena in a \textit{principled} and \textit{unified} way?
   \item The \textbf{Bayesian framework} is a good way to do that.
    \end{itemize}
  \end{itemize}
\end{frame}


\begin{frame}{Session 3 – Overview}
  \begin{itemize}
    \item \textbf{What we’ll explore:}
    \begin{itemize}
      \item \textbf{The Bayesian Framework}
      \item \textbf{Key Ingredients:}
      \begin{itemize}
        \item \textbf{Prior Distribution:} Our belief before seeing the data.
        \item \textbf{Likelihood:} How compatible is the observed data is with different parameter values.
        \item \textbf{Posterior Distribution:} Our updated beliefs after observing the data.
        \item \textbf{Predictive Distribution:} Make predictions about new, unseen data.
      \end{itemize}
    \end{itemize}

    \item \textbf{What you should already know:}
    \begin{itemize}
      \item The intuition behind probabilistic modeling — Alzheimer’s test case (Session 1)
      \item Core probability rules: sum, product, and Bayes' rule (Session 1)
      \item Handle multivariate distributions and apply their properties (Session 2)
    \end{itemize}
  \end{itemize}
\end{frame}

\section{Properties of a Probability Distribution}

\section{Important Distributions}


\end{document}
